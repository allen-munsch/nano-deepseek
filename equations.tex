\documentclass{article}
\usepackage{amsmath}
\usepackage{amssymb}
\usepackage{physics}
\usepackage{bm}
\usepackage{natbib}

\title{Mathematical Formulation of DeepSeek with Probabilistic Extensions}
\author{DeepSeek Implementation}
\date{\today}

\begin{document}
\maketitle

\begin{abstract}
This document presents the mathematical formulation of a probabilistic language model implementation. The approach combines classical deep learning with stochastic computing techniques, including Monte Carlo methods, particle-based optimization, and uncertainty estimation. The implementation focuses on robust probabilistic modeling while maintaining computational efficiency.
\end{abstract}

\section*{Introduction}
This document describes the mathematical formulation of a quantum-enhanced language model implementation. The codebase combines classical transformer architecture with quantum computing concepts, including quantum layers for processing, quantum-inspired optimization, and Monte Carlo methods for uncertainty estimation. Key components include quantum state evolution through rotation gates, error correction mechanisms, and quantum-enhanced attention mechanisms. The implementation focuses on maintaining quantum coherence while providing practical benefits for language modeling tasks.

\section{Model Architecture}

\subsection{Quantum State Preparation and Evolution}
The quantum layer implements proper quantum state preparation and evolution:

\subsubsection{Amplitude Encoding}
Classical to quantum state conversion with proper amplitude encoding:
\begin{equation}
|\psi_{\text{in}}\rangle = \sum_{i=0}^{2^{\lceil\log_2(n)\rceil}-1} r_i e^{i\theta_i}|i\rangle
\end{equation}

where amplitudes $\alpha_i$ are computed as:
\begin{equation}
\alpha_i = \text{sign}(x_i)\frac{|x_i|}{\|\mathbf{x}\|_2}e^{i\phi_i}, \quad \phi_i = \arccos(|x_i|/\|\mathbf{x}\|_2)
\end{equation}

ensuring proper normalization:
\begin{equation}
\sum_i |\alpha_i|^2 = 1
\end{equation}

\subsubsection{Quantum Gates}
The circuit applies a sequence of quantum gates:
\begin{align*}
H &= \frac{1}{\sqrt{2}}\begin{pmatrix} 1 & 1 \\ 1 & -1 \end{pmatrix} \text{ (Hadamard)} \\
R_x(\theta) &= \begin{pmatrix} \cos(\theta/2) & -i\sin(\theta/2) \\ -i\sin(\theta/2) & \cos(\theta/2) \end{pmatrix} \text{ (X-rotation)} \\
\text{CNOT} &= |0\rangle\langle0| \otimes I + |1\rangle\langle1| \otimes X \text{ (Entanglement)}
\end{align*}

\subsubsection{Quantum Evolution}
The quantum state evolves through the circuit:
\begin{equation}
|\psi_{\text{out}}\rangle = U_M\cdots U_2U_1|\psi_{\text{in}}\rangle
\end{equation}

where $U_i$ are unitary quantum operations including single-qubit gates and two-qubit entangling gates.

\subsection{Monte Carlo Attention}
The Monte Carlo attention mechanism combines multiple sampling strategies:
\begin{align*}
\text{logits}_{\text{gumbel}} &= \frac{\text{logits} + \mathcal{G}}{0.1} \\
\text{logits}_{\text{temp}} &= \frac{\text{logits}}{0.8} \\
\text{mask}_{\text{top-k}} &= \text{TopK}(\text{logits}, k=40) \\
\text{logits}_{\text{combined}} &= (\text{logits}_{\text{gumbel}} + \text{logits}_{\text{temp}}) \odot \text{mask}_{\text{top-k}} \\
\text{probs} &= \text{softmax}(\text{logits}_{\text{combined}}) \\
\text{probs}_{\text{noisy}} &= \text{softmax}(\log(\text{probs} + \epsilon) + \mathcal{N}_{\text{explore}})
\end{align*}

where $\mathcal{G}$ is Gumbel noise, $\mathcal{N}_{\text{explore}} \sim \mathcal{N}(0, 0.05)$ is exploration noise, and $\epsilon=10^{-10}$.

\subsection{Rotary Embeddings}
Position-dependent rotation matrices applied to queries and keys:
\begin{equation}
\begin{split}
q_{\text{rot}} &= q \odot \text{PE}(L,d) \\
k_{\text{rot}} &= k \odot \text{PE}(L,d)
\end{split}
\end{equation}

where $\text{PE}$ is the positional encoding with base=10000.

\section{Loss Functions}

\subsection{Combined Loss}
The total loss combines cross entropy, KL divergence, and uncertainty:
\begin{equation}
\mathcal{L}_{\text{total}} = \frac{1}{\text{grad\_accum}}\left(\mathcal{L}_{\text{CE}} + 0.1\mathcal{L}_{\text{KL}} + 0.1\mathcal{U} + \mathcal{D}\right)
\end{equation}

where:
\begin{align*}
\mathcal{L}_{\text{CE}} &= -\sum_{i} y_i \log(\hat{y}_i) \\
\mathcal{L}_{\text{KL}} &= \text{KL}(\log(\text{softmax}(\text{logits})) \| \text{probs}) \\
\mathcal{U} &= \text{std}(\{\mathcal{L}_i\}_{i=1}^{N_{\text{MC}}}) \\
\mathcal{D} &= -0.1 \cdot \text{std}(\{\hat{y}_i\}_{i=1}^{N_{\text{MC}}})
\end{align*}

\subsection{Quantum Coherence}
The von Neumann entropy for quantum coherence:
\begin{equation}
S(\rho) = -\sum_i \lambda_i \log(\lambda_i + \epsilon)
\end{equation}

where $\lambda_i$ are eigenvalues of the density matrix $\rho$ and $\epsilon=10^{-10}$.

\section{Optimization}

\subsection{Quantum Adam with High-Dimensional Processing}
The quantum-enhanced Adam optimizer leveraging high-dimensional quantum operations:
\begin{align*}
m_t &= \beta_1 m_{t-1} + (1-\beta_1)g_t \\
v_t &= \beta_2 v_{t-1} + (1-\beta_2)g_t^2 \\
|\psi_t\rangle &= \mathcal{QFT}(e^{i p_t}) \\
\text{Phase}_t &= \text{diag}(\mathcal{U}_t^\dagger \mathcal{U}_t) \\
\text{Energy}_t &= |\langle\psi_t|\mathcal{H}|\psi_t\rangle|^2 \\
q_t &= q_{\text{factor}} \sin(\text{Phase}_t) \sqrt{\text{Energy}_t} \\
m^q_t &= \beta_q m^q_{t-1} + (1-\beta_q)\mathcal{QFT}^{-1}(q_t) \\
p_{t+1} &= p_t - \alpha\frac{m_t}{\sqrt{v_t} + \epsilon} + m^q_t
\end{align*}

where $\mathcal{QFT}$ enables parallel quantum operations, $\mathcal{U}_t$ is the time evolution operator, $\mathcal{H}$ is the system Hamiltonian, and $\beta_q=0.9$ controls quantum momentum. This formulation achieves $O(n\log n)$ complexity through quantum parallelism.

\subsection{Learning Rate Schedule}
Cosine decay with linear warmup:
\begin{equation}
\eta_t = \begin{cases}
\eta_{\text{max}}\frac{t}{t_{\text{warmup}}} & \text{if } t < t_{\text{warmup}} \\
\eta_{\text{min}} + \frac{\eta_{\text{max}}-\eta_{\text{min}}}{2}(1 + \cos(\pi\frac{t-t_{\text{warmup}}}{t_{\text{max}}-t_{\text{warmup}}})) & \text{otherwise}
\end{cases}
\end{equation}

where $\eta_{\text{min}}=3\times10^{-5}$ and $\eta_{\text{max}}=3\times10^{-4}$.

\section{Error Correction and Noise}

\subsection{Surface Code Error Correction}
The Surface code error correction process:
\begin{align*}
|\psi_{\text{encoded}}\rangle &= \mathcal{S}_d|\psi_{\text{in}}\rangle \\
|\psi_{\text{noisy}}\rangle &= \mathcal{N}|\psi_{\text{encoded}}\rangle \\
\text{syndromes} &= \{s_p, s_v\} = \{\langle \psi_{\text{noisy}}|P_i|\psi_{\text{noisy}}\rangle, \langle \psi_{\text{noisy}}|V_j|\psi_{\text{noisy}}\rangle\} \\
\mathcal{C} &= \text{MWPM}(\text{syndromes}) \\
|\psi_{\text{corrected}}\rangle &= \prod_{c \in \mathcal{C}} X_c Z_c |\psi_{\text{noisy}}\rangle \\
|\psi_{\text{out}}\rangle &= \mathcal{S}_d^\dagger|\psi_{\text{corrected}}\rangle
\end{align*}

where $\mathcal{S}_d$ is the Surface code encoding for distance $d$, $P_i$ and $V_j$ are plaquette and vertex stabilizer operators, MWPM is minimum weight perfect matching, and $X_c, Z_c$ are Pauli corrections along correction chains $\mathcal{C}$.

\subsection{Noise Model}
The quantum noise simulation:
\begin{equation}
|\psi_{\text{noisy}}\rangle = \frac{(1 + \mathcal{N}_{\text{amp}})\exp(i\mathcal{N}_{\text{phase}})|\psi\rangle}{\|(1 + \mathcal{N}_{\text{amp}})\exp(i\mathcal{N}_{\text{phase}})|\psi\rangle\|}
\end{equation}

where $\mathcal{N}_{\text{amp}} \sim \mathcal{N}(0,0.01)$ and $\mathcal{N}_{\text{phase}} \sim \mathcal{N}(0,0.005)$.

\section{Monte Carlo Methods}

\subsection{Sampling Strategy}
Multiple sampling approaches combined:
\begin{equation}
\text{logits}_{\text{combined}} = (\text{logits}_{\text{gumbel}} + \text{logits}_{\text{temp}}) \odot \text{mask}_{\text{top-k}}
\end{equation}

where $\text{mask}_{\text{top-k}}$ selects top 40 logits.

\subsection{Uncertainty Estimation}
Mean and uncertainty from MC samples:
\begin{align*}
\bar{y} &= \frac{1}{N_{\text{MC}}}\sum_{i=1}^{N_{\text{MC}}} \hat{y}_i \\
\sigma^2 &= \frac{1}{N_{\text{MC}}}\sum_{i=1}^{N_{\text{MC}}}(\hat{y}_i - \bar{y})^2
\end{align*}

with $N_{\text{MC}}=8$ samples for evaluation.

\bibliographystyle{plainnat}
\bibliography{references}

\end{document}
