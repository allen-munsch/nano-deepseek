\documentclass{article}
\usepackage{amsmath}
\usepackage{amssymb}
\usepackage{physics}
\usepackage{bm}

\title{Mathematical Formulation of DeepSeek with Quantum Enhancement}
\author{DeepSeek Implementation}
\date{\today}

\begin{document}
\maketitle

\section*{Introduction}
This document describes the mathematical formulation of a quantum-enhanced language model implementation. The codebase combines classical transformer architecture with quantum computing concepts, including quantum layers for processing, quantum-inspired optimization, and Monte Carlo methods for uncertainty estimation. Key components include quantum state evolution through rotation gates, error correction mechanisms, and quantum-enhanced attention mechanisms. The implementation focuses on maintaining quantum coherence while providing practical benefits for language modeling tasks.

\section{Model Architecture}

\subsection{Quantum Layer Operations}
The quantum layer applies a series of rotation gates and error correction:
\begin{equation}
|\psi_{\text{out}}\rangle = \mathcal{E}(\mathcal{R}_z(\theta_z)\mathcal{R}_y(\theta_y)\mathcal{R}_x(\theta_x))|\psi_{\text{in}}\rangle
\end{equation}

where $\mathcal{R}_i$ represents rotation gates and $\mathcal{E}$ is the error correction operator.

\subsection{Monte Carlo Attention}
The Monte Carlo attention mechanism with temperature scaling and Gumbel noise:
\begin{equation}
\text{Attention}(Q,K,V) = \frac{1}{N_{\text{MC}}}\sum_{i=1}^{N_{\text{MC}}} \text{softmax}\left(\frac{QK^T}{\sqrt{d}} + \mathcal{G}_i\tau\right)V
\end{equation}

where $\tau=0.6$ is the temperature parameter and $\mathcal{G}_i$ is Gumbel noise.

\subsection{Rotary Embeddings}
Position-dependent rotation matrices applied to queries and keys:
\begin{equation}
\begin{split}
q_{\text{rot}} &= q \odot \text{PE}(L,d) \\
k_{\text{rot}} &= k \odot \text{PE}(L,d)
\end{split}
\end{equation}

where $\text{PE}$ is the positional encoding with base=10000.

\section{Loss Functions}

\subsection{Combined Loss}
The total loss combines cross entropy, uncertainty, and quantum coherence:
\begin{equation}
\mathcal{L}_{\text{total}} = \frac{1}{\text{grad\_accum}}\left(\mathcal{L}_{\text{CE}} + 0.1\mathcal{U} + \mathcal{D} + 0.01\mathcal{L}_{\text{quantum}}\right)
\end{equation}

where:
\begin{align*}
\mathcal{L}_{\text{CE}} &= -\sum_{i} y_i \log(\hat{y}_i) \\
\mathcal{U} &= \text{std}(\{\mathcal{L}_i\}_{i=1}^{N_{\text{MC}}}) \\
\mathcal{D} &= -0.1 \cdot \text{std}(\{\hat{y}_i\}_{i=1}^{N_{\text{MC}}}) \\
\mathcal{L}_{\text{quantum}} &= -\text{Tr}(\rho\log\rho)
\end{align*}

\subsection{Quantum Coherence}
The von Neumann entropy for quantum coherence:
\begin{equation}
S(\rho) = -\sum_i \lambda_i \log(\lambda_i + \epsilon)
\end{equation}

where $\lambda_i$ are eigenvalues of the density matrix $\rho$ and $\epsilon=10^{-10}$.

\section{Optimization}

\subsection{Quantum Adam}
The quantum-enhanced Adam optimizer with quantum momentum:
\begin{align*}
m_t &= \beta_1 m_{t-1} + (1-\beta_1)g_t \\
v_t &= \beta_2 v_{t-1} + (1-\beta_2)g_t^2 \\
\phi_t &= \angle(e^{i\pi p_t}) \\
q_t &= q_{\text{factor}} \sin(\phi_t) m_{\text{quantum}} \\
p_{t+1} &= p_t - \alpha\frac{m_t}{\sqrt{v_t} + \epsilon} + q_t
\end{align*}

where $q_{\text{factor}}=0.1$ controls quantum effects.

\subsection{Learning Rate Schedule}
Cosine decay with linear warmup:
\begin{equation}
\eta_t = \begin{cases}
\eta_{\text{max}}\frac{t}{t_{\text{warmup}}} & \text{if } t < t_{\text{warmup}} \\
\eta_{\text{min}} + \frac{\eta_{\text{max}}-\eta_{\text{min}}}{2}(1 + \cos(\pi\frac{t-t_{\text{warmup}}}{t_{\text{max}}-t_{\text{warmup}}})) & \text{otherwise}
\end{cases}
\end{equation}

where $\eta_{\text{min}}=3\times10^{-5}$ and $\eta_{\text{max}}=3\times10^{-4}$.

\section{Error Correction and Noise}

\subsection{Error Correction}
The error correction process:
\begin{align*}
|\psi_{\text{encoded}}\rangle &= \mathcal{C}|\psi_{\text{in}}\rangle \\
\text{syndrome} &= \mathcal{M}|\psi_{\text{encoded}}\rangle \\
|\psi_{\text{corrected}}\rangle &= \begin{cases}
|\psi_{\text{encoded}}\rangle_{\text{flip}} & \text{if syndrome} > 0 \\
|\psi_{\text{encoded}}\rangle & \text{otherwise}
\end{cases}
\end{align*}

where $\mathcal{C}$ is the encoding matrix and $\mathcal{M}$ is the syndrome measurement.

\subsection{Noise Model}
The quantum noise simulation:
\begin{equation}
|\psi_{\text{noisy}}\rangle = \frac{(1 + \mathcal{N}_{\text{amp}})\exp(i\mathcal{N}_{\text{phase}})|\psi\rangle}{\|(1 + \mathcal{N}_{\text{amp}})\exp(i\mathcal{N}_{\text{phase}})|\psi\rangle\|}
\end{equation}

where $\mathcal{N}_{\text{amp}} \sim \mathcal{N}(0,0.01)$ and $\mathcal{N}_{\text{phase}} \sim \mathcal{N}(0,0.005)$.

\section{Monte Carlo Methods}

\subsection{Sampling Strategy}
Multiple sampling approaches combined:
\begin{equation}
\text{logits}_{\text{combined}} = (\text{logits}_{\text{gumbel}} + \text{logits}_{\text{temp}}) \odot \text{mask}_{\text{top-k}}
\end{equation}

where $\text{mask}_{\text{top-k}}$ selects top 40 logits.

\subsection{Uncertainty Estimation}
Mean and uncertainty from MC samples:
\begin{align*}
\bar{y} &= \frac{1}{N_{\text{MC}}}\sum_{i=1}^{N_{\text{MC}}} \hat{y}_i \\
\sigma^2 &= \frac{1}{N_{\text{MC}}}\sum_{i=1}^{N_{\text{MC}}}(\hat{y}_i - \bar{y})^2
\end{align*}

with $N_{\text{MC}}=8$ samples for evaluation.

\end{document}
